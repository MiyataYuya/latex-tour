\documentclass[a4paper]{ltjsarticle}
\usepackage{packages}

\author{著者}
\date{\和暦\today}
\title{タイトル}

% 好きなフォントを使うにはこんな風に
\newfontfamily\shoujo[%
    Path = fonts/,
    UprightFont = *-p7PAK,
    Extension = .ttf
    ]{Nkf10MagicumComicumCrassum}

\begin{document}
\maketitle
\setcounter{tocdepth}{3}
\tableofcontents
%
\begin{abstract}
    アブストラクトはこんなふうに。
\end{abstract}
%
%
\section{目次にサブサブセクションまで追加するには}

標準ではサブセクションまでしか表示されない。サブサブセクションまで表示したい場合は、\verb|\setcounter{tocdepth}{n}|を加える。
\subsection{サブセクション}
\subsubsection{サブサブセクション}
\section{フォントを細かく変える}
フォントの宣言はプリアンブルで行う。例えば\verb|./fonts/Nkf10MagicumComicumCrassum-p7PAK.ttf|から新たなフォントファミリー\verb|\shoujo|を定義するには次のようにする。
\begin{verbatim}
\newfontfamily\shoujo[%
    Path = fonts/,
    UprightFont = *-p7PAK,
    Extension = .ttf
    ]{Nkf10MagicumComicumCrassum}
\end{verbatim}
すると、
{\shoujo We the People of the United States, in Order to form a more perfect Union, establish Justice, insure domestic Tranquility, provide for the common defence, promote the general Welfare, and secure the Blessings of Liberty to ourselves and our Posterity, do ordain and establish this Constitution for the United States of America.}

(We the People of the United States, in Order to form a more perfect Union, establish Justice, insure domestic Tranquility, provide for the common defence, promote the general Welfare, and secure the Blessings of Liberty to ourselves and our Posterity, do ordain and establish this Constitution for the United States of America\cite{UScontrib}.)

\section{図の入れ方}
可換図式はtikz-cdを使う。マニュアルを見たい場合はターミナルから\verb|texdoc tikzcd|を実行。
\begin{mdframed}
\centering
\begin{tikzcd}[]
    A \arrow[r, "\phi"] \arrow[d, red] & B \arrow[d,"\psi" red] \\
    C \arrow[r,red,"\eta" blue]        & |[blue, rotate=345]|D
\end{tikzcd}
\end{mdframed}
位置はずれないみたい。背景にこんな風に色をつけるには、\verb|mdframed|パッケージを使う。

\section{ファイルの分割}
ファイルが巨大になる場合も簡単に分割できる。
\documentclass{ltjsarticle}

\usepackage{packages}
\begin{document}
``※この文章は分割ファイル※"
\end{document}
※改ページや改行はされない。
%
\addcontentsline{toc}{section}{参考文献}
\bibliographystyle{plain}
\bibliography{ref}
\end{document}
