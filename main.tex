\documentclass{ltjsarticle}
\author{Unko Bempi}
\date{\和暦\today}
\title{Unko ---unko---}

\newcount\it
\def\myfor#1{
    \it = 0 \loop\ifnum\it < #1
    ようこそ、うんこ。\\
    \advance\it by1\repeat
}


\begin{document}
\maketitle
\tableofcontents
%
\begin{abstract}
    スカトロジー(英語: Scatology)とは、糞・尿に対する研究・考察を言い、日本語では糞便学(ふんべんがく)とも言う。また、糞尿への興味や、それを愛玩する性的興奮、性癖・性的嗜好、糞尿に関するユーモア、糞尿や糞尿愛好等を主題とした文芸作品等も指す。糞尿趣味を指す場合はスカトロと略されることもある。これらに関係する人間を指してスカトロジストと呼ぶ\cite{wiki}。
    こんにちは、うんこ。
\end{abstract}
%
うんこは<うんこ>を表す。うんこ自体が<うんこ>であるわけではなく、うんこはあくまで<うんこ>を指し示す表現でしかない。
    \myfor{3}
%
\section{lualatex}
\newcommand{\texcmd}{some txtxtxtxtxtxt}
%
lualatexは以下のようにluaを扱えることが特徴の一つみたいだが、私には旨みがわからない。printのフォーマット方法が不明なので、九九の表も見るに耐えない出来である。
%
\begin{center}
    $ $
    \directlua{
        tex.print('\string\\\\')
        for i = 1, 9 do
            for j = 1, 9 do
                tex.print(i*j .. ' ')
            end
            tex.print('\string\\\\')
        end
    }
\end{center}
\section{うんこについて}
ここでは、うんこについての概要を書く。

\section{「このクソッタレが!」}
「このクソッタレが!」名古屋の高級焼肉店個室で“人糞”放置事件が発生 県議らの会食後に一体何が?《店のオーナーが“憤怒の告発”》


\section{岡﨑は書けるのか?}
lualatexを使えば、「﨑」の字をそのまま表示することができる。彁\cite{yurei}もできる。






\begin{thebibliography}{99}
    \bibitem[1]{wiki} スカトロジー, ウィキペディア
    \bibitem[2]{yurei} http://www.asahi.com/special/kotoba/archive2015/moji/2011082400019.html
\end{thebibliography}
\end{document}
